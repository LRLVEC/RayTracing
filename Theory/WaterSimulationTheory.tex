\documentclass[UTF8]{ctexart}
\usepackage{geometry}
\usepackage{indentfirst}
\usepackage{hyperref}
\usepackage{harpoon}
\usepackage{amsmath}
\usepackage{amssymb}
\usepackage{graphicx}

\geometry{a4paper, left=1cm, right=1cm, top=2cm, bottom=2cm}
\setlength{\parindent}{1cm}
\renewcommand
\contentsname{Content}
\title{水面模拟算法理论}
\author{段元兴}
\date{\today}
\begin{document}
    \maketitle
    \thispagestyle{empty}
    \setcounter{page}{1}
    \newpage
    \section{公式推导}
        \indent 在简化近似下,水面的波动可以近似为一个二维弹性膜的振动.于是可以由二维波动
        方程描述:
        \begin{equation}
            \frac{\partial^2{u}}{\partial{t}^2}=c^2\nabla^2u.
        \end{equation}
        在笛卡尔直角坐标系下可以写作:
        \begin{equation}
            \frac{\partial^2{u}}{\partial{t}^2}=c^2\left(\frac{\partial^2{u}}{\partial{x}^2}+
                \frac{\partial^2{u}}{\partial{y}^2}\right).
        \end{equation}
        而对于我们的正方形游泳池模型,可以得到如下边界条件:
        \begin{equation}
            \left\{
                \begin{array}{l}
                    \left.\frac{\partial^2{u}}{\partial{t}^2}-c^2\frac{\partial{u}}{\partial{x}}\right|_{x=0}=0\\
                    \left.\frac{\partial^2{u}}{\partial{t}^2}+c^2\frac{\partial{u}}{\partial{x}}\right|_{x=a}=0\\
                    \left.\frac{\partial^2{u}}{\partial{t}^2}-c^2\frac{\partial{u}}{\partial{y}}\right|_{y=0}=0\\
                    \left.\frac{\partial^2{u}}{\partial{t}^2}+c^2\frac{\partial{u}}{\partial{y}}\right|_{y=a}=0
                \end{array}
            \right..
        \end{equation}
        将方程差分化可得:
        \begin{equation}
            \left\{
                \begin{array}{l}
                    a_{ijk}=\frac{c^2}{\left(\Delta a\right)^2}*
                        \left\{
                            \begin{array}{lcc}
                                u_{i+1,jk}+u_{i-1,jk}+u_{i,j+1,k}+u_{i,j-1,k}-4u_{ijk}& 0<i<n_x-1,&0<j<n_y-1\\
                                u_{i+1,jk}+u_{i,j+1,k}+u_{i,j-1,k}-3u_{ijk}& i=0,&0<j<n_y-1\\
                                u_{i-1,jk}+u_{i,j+1,k}+u_{i,j-1,k}-3u_{ijk}& i=n_x-1,&0<j<n_y-1\\
                                u_{i,j+1,k}+u_{i+1,jk}+u_{i-1,jk}-3u_{ijk}& 0<i<n_x-1,&j=0\\
                                u_{i,j-1,k}+u_{i+1,jk}+u_{i-1,jk}-3u_{ijk}& 0<i<n_x-1,&j=n_y-1\\
                                u_{i+1,jk}+u_{i,j+1,k}-2u_{ijk}& i=0,&j=0\\
                                u_{i+1,jk}+u_{i,j-1,k}-2u_{ijk}& i=0,&j=n_y-1\\
                                u_{i-1,jk}+u_{i,j+1,k}-2u_{ijk}& i=n_x-1,&j=0\\
                                u_{i-1,jk}+u_{i,j-1,k}-2u_{ijk}& i=n_x-1,&j=n_y-1
                            \end{array}
                        \right.\\
                    u_{ij,k+1}=v_{ijk}\Delta t +\frac{\left(\Delta t\right)^2}{2}a_{ijk}\\
                    v_{ij,k+1}=v_{ijk}+a_{ijk}\Delta t\\
                \end{array}
            \right..
        \end{equation}
        最后将方程进行多次迭代就能近似解出水面的波动形式.
    \section{与光线追踪算法结合}
        \indent 为了将这些点渲染出来,首先将这些点相互连接形成许多三角形,然后将其渲染出来就行了.
        传统的光栅化算法当然十分高效,基本可以忽略三角形绘制所需的时间,但是这样得到的画质也是惨不忍睹的.
        因此我们采用光线追踪算法来渲染.由于在CPU上构建BVH需要耗费大量时间,这与所需要的高性能是相悖的,
        考虑到水面有一定的波动范围,因此在构建BVH的时候就将其z方向(振动方向)设为该阈值,这样就无需重复构造
        没有必要的包围盒.而三角形的预计算本来就是在GPU上以并行化的方式进行,基本不需要担心这一部分的计算速度.
        而且可以判断哪些三角形面片是属于水面的,在三角形预计算的时候判断一下就能将静态的三角形剔除从而加速.
\end{document}